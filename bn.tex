\section{BAYESIAN NETWORKS}
\label{sec:bns}

Bayesian networks are a general way to describe the dependencies
between the components of an $n$\nobreakdash-dimensional discrete data
vector $X=(X_{1},\ldots,X_{n})$ in which the component $X_{i}$ may
take any of the discrete values in a set $\{1,\ldots,r_{i}\}$.
Despite denoting the values with small integers, the model will treat
the components of $X$ categorical.


\subsection{Likelihood}
\label{ssec:likelihood}

Bayesian network $B=(G,\theta)$ defines a probability distribution for
$X$. The component $G$ defines the structure of the model as a
directed acyclic graph that has exactly one node for each component of
$X$. The structure $G=(G_{1},\ldots,G_{n})$ defines for each
variable/node $X_{i}$ its (possibly empty) parent set $G_{i}$, i.e.,
the nodes from which there are a directed edges to the variable
$X_{i}$.

Given a realization $x$ of $X$, we denote the sub\nobreakdash-vector
of $x$ that consists of the values of the parents of $X_{i}$ in $x$ as
$G_{i}(x)$. It is customary to enumerate all the possible
sub\nobreakdash-vectors $G_{i}(X)$ from $1$ to $q_{i}=\prod_{h\in
  G_{i}}r_{h}.$ In case $G_{i}$ is empty, we define $q_{i}=1$ and
$P(G_{i}(x)=1)=1$ for all vectors $x$.

For each variable $X_{i}$ there is a $q_{i}\times r_{i}$ table
$\theta_{i}$ of parameters whose $j^{\textnormal{th}}$ row and
$k^{\textnormal{th}}$ column defines the conditional probability
$P(X_{i}=k\mid G_{i}(X)=j;\theta)=\theta_{ijk}$.  With structure $G$
and parameters $\theta$, we can now express the likelihood function of
the model as
\begin{equation}
P(x|G,\theta)=\prod_{i=1}^{n}P(x_{i}\mid G_{i}(x);\theta_{i})=\prod_{i=1}^{n}\theta_{iG_{i}(x)x_{i}}.
\end{equation}



\subsection{Bayesian structure learning}

Bayesian learning of Bayesian network structures is based on the
posterior probability $P(G|D,\alpha)$, where $\alpha$ denotes the
hyper\nobreakdash-parameters for the model parameters $\theta$, and
the $D$ is a collection of $N$ $n$\nobreakdash-dimensional i.i.d. data
vectors collected to a $N\times n$ design matrix.  We use structure
$G$ to extend the common matrix indexing notation $D[i,j]$, and allow
more general row selectors.  Notably the notation $D[G_i=j,i]$ is used
to denote those rows of the column $i$ in which the parents $G_i$
contain the value configuration number $j\in\{1,\ldots,q_i\}$.

It is common to assume the uniform prior for structures, in which case
the objective function for structure learning is reduced to the
marginal likelihood $P(D|G,\alpha)$.  If the model parameters
$\theta_{ij}$ are further assumed to be independently Dirichlet
distributed only depending on $i$ and $G_{i}$ and the data $D$ is
assumed to have no missing values, the marginal likelihood can be
decomposed as
\begin{eqnarray}
\label{eqn:bayesmix}
P(D|G,\alpha) & = & \prod_{i=1}^{n}\prod_{j=1}^{q_i}\int
P(D[G_i=j,i]|\alpha)\\ & = & \prod_{i=1}^{n}\prod_{j=1}^{q_i}\int
P([G_i=j,i]|\theta)P(\theta|\alpha) d\theta.\nonumber
\end{eqnarray}
In coding terms this means that each data column $D[\cdot,i]$ is first partitioned based on the
values in columns $G_i$, and each part is then coded using a Bayesian mixture that can be expressed
in a closed form~\cite{Bunt91, Heck95}.

\subsection {Problems, solutions and problems}

Finding the satisfactory Dirichlet hyperparameters for the Bayesian
mixture above has, however, turned out to be problematic. Early on,
one of the desiderata for a good model selection criterion was that it
would yield equal scores for essentially equivalent
models~\cite{Verm90}.  For example, the score for the structure
$X_1\rightarrow X_2$ should be the same as the score for the model
$X_2 \rightarrow X_1$ since they both correspond to the hypothesis
that variables $X_1$ and $X_2$ are statistically dependent on each
other.  It can be shown~\cite{Heck95} that to achieve this, not all
the hyperparameters $\alpha$ are possible and for practical reasons
Buntine~\cite{Bunt91} suggested a so called BDeu score with just one
hyperparameter $\alpha\in R_{++}$ so that $\theta_{ij\cdot}\sim
Dir(\frac{\alpha}{q_i r_i},\ldots,\frac{\alpha}{q_i r_i})$.  However,
it soon turned out that BDeu score was very sensitive to the selection
of this hyperparameter~\cite{cosco.uai07} and that for small sample
sizes this method detects spurious correlations~\cite{Steck08} leading
to models with suspiciously many parameters. 

Recently, Suzuki~\cite{Suzuki2017} discussed the theoretical properties of the BDeu of score and showed that in certain settings BDeu has tendency to add more and more parent variables for a child node even though the empirical conditional entropy of the child given the parents has already reached zero. More in detail, assume that in our data $D$, the values of $X_i$ are completely determined by variables in set $Z$, so that the entropy $H(X | Z) = 0$. Now, if we can further find one or more variables, denoted by $Y$, whose values are determined completely by the variables in $Z$, then BDeu will prefer the set $Z\cup Y$ over mere $Z$ as the parents of $X_i$. Suzuki argues that this kind of behaviour violates regularity in model selection as the more complex model is preferred over a simpler one even though it does not fit to data any better. The phenomena seems to stem from the way the hyperparameters for the Dirichlet distribution are chosen in BDeu as the  BD score based on the Jeffreys prior does not suffer from this anomaly.    

A natural solution to avoid parameter sensitivity of BDeu would be to
use a normalized maximum likelihood (NML)
criterion~\cite{Shta87,Riss96a}, i.e., to find the structure $G$ that
maximizes
\begin{equation}
P_{NML}(D;G)=\frac{P(D|\hat\theta(D;G))}{\sum_{D'}{P(D'|\hat\theta(D';G))}},
\end{equation}
where $\hat\theta$ denotes the (easy to find) maximum likelihood
parameters and the sum in the denominator goes over all the possible
$N\times n$ data matrices. While it is easy to see that this criterion
satisfies the requirement of giving equal scores to equal structures,
the normalizing constant renders the computation
infeasible. Consequently, Silander et al.~\cite{cosco.pgm08a}
suggested solving the parameter sensitivity problem by adopting the
NML-code to the column partitions, i.e., changing the Bayesian mixture
in equation~(\ref{eqn:bayesmix}) to
\begin{equation}
P^1_{NML}(D[G_i=j,i];G)=\frac{P(D|\hat\theta(D[G_i=j,i];G))}{\sum_{D'}{P(D';\hat\theta(D'|G))}},
\end{equation}
where $D'\in{\{1,\ldots,r_i\}}^{|D[G_i=j,i]|}$.  The logarithm of the
denominator is often called a regret, since it indicates the extra
code length needed compared to the code length obtained using the (a
priori unknown) maximum likelihood parameters. The regret for
$P^1_{NML}$ depends only on the length $N$ of the categorical data
vector with $r$ different categorical values.  While the naive
computation of the regret is still prohibitive, it can be approximated
efficiently using an approximation derived by Szpankowski and
Weinberger~\cite{Szpankowski2012}:
\begin{eqnarray}
    reg(N, r) & \approx & N\left(\log{\alpha} + (\alpha + 2) \log{C_\alpha}
                - \frac{1}{C_\alpha}\right)\nonumber \\
    && - \frac{1}{2} \log{\left(C_\alpha + \frac{2}{\alpha}\right)},
\end{eqnarray}
where $\alpha = \frac{r}{N}$ and
\begin{equation}
    C_\alpha = \frac{1}{2} + \frac{1}{2} \sqrt{1 + \frac{4}{\alpha}}.
\end{equation}
fNML solves the parameter sensitivity problem and yields predictive
models superior to BDeu.  However, the criterion does not satisfy the
property of giving the same score for the models that correspond to
the same dependence statements. Furthermore, the learned structures
still sometimes appear surprisingly complicated.
