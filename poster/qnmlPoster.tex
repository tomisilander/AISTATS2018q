\documentclass[final]{beamer}

\mode<presentation>
{
  \usetheme{hiitposter}
}
%\graphicspath{{imgs/}}
\usepackage{amsmath,amsthm, amssymb,mathrsfs}
\usepackage{color}
\usepackage{times}
\usepackage{subfigure}
\usepackage{bm}
\usepackage{url}
%\usepackage[orientation=portrait,size=a0,scale=1.4,debug]{beamerposter}        
\usepackage[orientation=portrait,width = 91 ,height = 122 ,scale=1.39,debug]{beamerposter} 
%\setbeamerfont{itemize/enumerate body}{size={\fontsize{25}{31}}}
%\setbeamerfont{itemize/enumerate subbody}{size=\footnotesize}

\graphicspath{{../qNML_images/}}

\def\newblock{\hskip .11em plus .33em minus 3.07em}

\definecolor{DRed}{rgb}{0.8,0.6,0.6}
\definecolor{DGreen}{rgb}{0.6,0.7,0.6}
\definecolor{BRed}{rgb}{0.7,0,0}
\definecolor{BGreen}{rgb}{0,0.7,0}

\newcommand{\mechanism}{\mathcal{M}}
\newcommand{\dataset}{\mathcal{D}}

\newcommand{\myred}[1]{{\color{BRed}#1}}
\newcommand{\mygreen}[1]{{\color{BGreen}#1}}


\newcommand{\heading}[1]{\alert{\large #1}\\}
\definecolor{myPurple}{RGB}{174, 206, 195}

\usepackage{textpos}
\usepackage{fancybox}
\usepackage{tikz}
\usepackage{mathtools}
\usepackage{empheq}

\theoremstyle{plain}
\newtheorem{assumption}{Assumption}

\newcommand{\vOmega}{\bm{\Omega}}
\newcommand{\bBeta}{\bm{\beta}}
\newcommand{\R}{\mathbb{R}}
\newcommand{\vA}{\bm{A}}
\newcommand{\vx}{\bm{x}}
\newcommand{\vy}{\bm{y}}
\definecolor{myblue}{rgb}{.85, .85, 1.0}

%%%%%%%%%%%%%%%%%%%%%%%%%%%%%%%%%%%%%%%%
\newlength\mytemplen
\newsavebox\mytempbox

\makeatletter
\newcommand\mybluebox{%
    \@ifnextchar[%]
       {\@mybluebox}%
       {\@mybluebox[0pt]}}

\def\@mybluebox[#1]{%
    \@ifnextchar[%]
       {\@@mybluebox[#1]}%
       {\@@mybluebox[#1][0pt]}}

\def\@@mybluebox[#1][#2]#3{
    \sbox\mytempbox{#3}%
    \mytemplen\ht\mytempbox
    \advance\mytemplen #1\relax
    \ht\mytempbox\mytemplen
    \mytemplen\dp\mytempbox
    \advance\mytemplen #2\relax
    \dp\mytempbox\mytemplen
    \colorbox{myblue}{\hspace{1em}\usebox{\mytempbox}\hspace{1em}}}

\makeatother
%%%%%%%%%%%%%%%%%%%%%%%%%%%%%%%%%%%%%%%%
\makeatletter
\newcommand\mathcircled[1]{%
  \mathpalette\@mathcircled{#1}%
}
\newcommand\@mathcircled[2]{%
  \tikz[baseline=(math.base)] \node[red,draw,circle,inner sep=1pt] (math) {$\m@th#1#2$};%
}
\makeatother


\title{Quotient Normalized Maximum Likelihood Criterion for Learning Bayesian Network Structures}
\author{Tomi Silander$^1$, Janne Lepp{\"a}-aho$^2$, Elias J{\"a}{\"a}saari$^2$, and Teemu Roos$^2$}
\institute{$^{1}$ NAVER LABS Europe, France \\ 
$^{2}$ HIIT / Department of Computer Science, University of Helsinki, Finland}

\begin{document}
\begin{frame}{}
\vskip-1.0ex

{
\setbeamercolor{block body}{bg=myPurple}
\begin{block}{Abstract}
	\large
  	We introduce an information theoretic criterion for Bayesian network
  	structure learning which we call quotient normalized maximum
	likelihood (qNML). In contrast to the closely related factorized
	normalized maximum likelihood criterion, qNML satisfies the property
	of score equivalence. It is also decomposable and completely free
	of adjustable hyperparameters. For practical computations, we identify
	a remarkably accurate approximation proposed earlier by Szpankowski
	and Weinberger. Experiments on both simulated and real data
	demonstrate that the new criterion leads to parsimonious models with
	good predictive accuracy.
\end{block}
}

\begin{block}{Section 1}
\begin{columns}[T]
   \begin{column}{0.3\textwidth} % first column {{{
     \heading{Subsection 1}
     
   \end{column}
  
   \begin{column}{0.005\textwidth}\linethickness{0.3ex} % separator {{{
      \color{myPurple} \line(0,1){400}
   \end{column} % }}}
   
   \begin{column}{0.3\textwidth} % second column {{{
     \heading{Subsection 2}
   \end{column}
   
   \begin{column}{0.005\textwidth}\linethickness{0.3ex} % separator {{{
      \color{myPurple} \line(0,1){400}
   \end{column} % }}}
   \begin{column}{0.3\textwidth}% third column {{{
     \heading{Subsection 3}
     
   \end{column} % end of first column }}}
\end{columns}
\end{block}


\begin{block}{Section 2}
\begin{columns}[T]
   \begin{column}{0.25\textwidth} % first column {{{
	\heading{Subsection 1}
     
     
   \end{column}
   
   \begin{column}{0.005\textwidth}\linethickness{0.3ex} % separator {{{
      \color{myPurple} \line(0,1){650}
   \end{column} % }}}
   
   \begin{column}{0.3\textwidth} % second column {{{
	\heading{Subsection 2}
	
   \end{column}
   
   \begin{column}{0.005\textwidth}\linethickness{0.3ex} % separator {{{
      \color{myPurple} \line(0,1){650}
   \end{column} % }}}
   
   
   \begin{column}{0.3\textwidth}% third column {{{
   
      \heading{Subsection 3}
   
   \end{column} % end of first column }}}
   % \begin{column}{0.005\textwidth}\linethickness{0.3ex} % separator {{{
   %    \color{myPurple} \line(0,1){600}
   % \end{column} % }}}
   % \begin{column}{0.3\textwidth} % third column {{{
   % \end{column} % end of third column }}}
   
\end{columns}
\end{block}

\begin{block}{Section 3}
  \begin{columns}[T]
    \begin{column}{0.35\textwidth}
      \heading{Subsection 1}

    \end{column}
   \begin{column}{0.005\textwidth}\linethickness{0.3ex}
      \color{myPurple} \line(0,1){800}
   \end{column} % }}}
   
    \begin{column}{0.23\textwidth}
    \heading{Subsection 2}
      
    \end{column}
    
    \begin{column}{0.005\textwidth}\linethickness{0.3ex} % separator {{{
      \color{myPurple} \line(0,1){800}
   \end{column} % }}}
    \begin{column}{0.3\textwidth}
    \heading{Subsection 3}
  
    \end{column}
  \end{columns}
\end{block}

\vskip-1.0ex
\begin{columns}[T]
  \begin{column}{0.63\paperwidth}
\begin{block}{References} %{{{
%\scriptsize
\tiny
{
[1] 
}
\end{block} %}}}
  \end{column}
  \begin{column}{0.33\paperwidth}
\begin{block}{Acknowledgements} %{{{
%\scriptsize
\tiny
{
..........
}
\end{block} %}}}
  \end{column}
\end{columns}
\end{frame}
\end{document}
 
