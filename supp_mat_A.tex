\section{Score equivalency proof}

Chickering showed that any equivalent structure can be reached from
another by a series of arc reversal operations without leaving the
equivalence class~\cite{Chick95}. This means that in order to show
that equivalent structures leads to the same qNML score, we only need
to prove that this is the case for equivalent structures that differ
by a single arc reversal. 

The network without any arcs is the sole member of its equivalence
class. All the other networks have at least one arc. We will first
presents some lemmas that characterize the relations of the parent sets
before and after the arc reversal. Let $G$ be a network structure
and $G'$ an equivalent network structure after the arc from $A$
to $B$ has been reversed to point from $B$ to $A$. We
will denote the parent sets of $A$ and $B$ in $G$ by $G_{A}$
and $G_{B}$, and the parent sets of $A$ and $B$ in $G'$ by $G'_{A}$
and $G'_{B}$. Note that not all the arc reversals lead to the equivalent
structures. By saying that $G$ and $G'$ are equivalent we imply
that our arc reversal does not destroy existing V-structures or create
new ones. The crucial observation is that if reversing the arc that
goes from $A$ to $B$ does not create or destroy V-structures, it
must be that the parents of $B$ other than $A$ are exactly the same
as parents of $A$ (statement 3 below). 

\begin{lemma}\label{thm:sameparents}
\item $G_{B}=G_{A}\cup\{A\}$
\end{lemma}

\begin{proof}
We show the direction $G_{A}\cup\{A\}\subset G_{B}$ 
by contradiction. First of all, a parent of $A$ cannot be a child
of $B$ or otherwise there would be a loop in $G.$ If $A$ had a
parent $Z$ that is not a parent of $B$, then $B$ and $Z$ were
not adjacent and the reversal of the arc would create a new V-structure
$(B,A,Z)$. Since this was forbidden by the equivalence of $G$ and
$G'$, $Z$ must also be a parent of $B$. 

Also, $B$ cannot have parents other than $A$ that are not also parents
of $A$, i.e., $G_{B}\subset G_{A}\cup{A}$ . If there were such
a parent $Z$ it should be anyway adjacent to $A$, otherwise the
V-structure $(A,B,Z)$ would be destroyed in the reversal. Since the
$Z$ was not a parent of $A$, the only possibility for adjacency
is that $Z$ were a child of $A$. However, in this case arc reversal
would lead to a loop $B,A,Z,B$.
\end{proof}

We will next list some simple consequences of the equivalence preserving
arc reversal:

\begin{lemma}
\label{thm:reveqs}
\hangindent\leftmargini
\hspace{1mm}
\begin{enumerate}
\item $G'_{B}=G_{B}\setminus\{A\}$
\item $G'_{A}=G_{A}\cup\{B\}$ 
\item $G'_{B}=G_{A}$
\item $\{B\}\cup G_{B}=\{B\}\cup G_{A}\cup\{A\}=G'_{A}\cup\{A\}$
\item $G'_{A}=G_{A}\cup\{B\}=G'_{B}\cup\{B\}$
\end{enumerate}
\end{lemma}


\begin{proof}

The statements 1 and 2 are trivial since they just state the fact
that an arc reversal removes $A$ from the parents of $B$ and adds
$B$ to the parents of $A$. 

The
statement 3 follows directly by using the lemma \ref{thm:sameparents}
to the statement 1. Combining these equalities we can generate more of them 
such as statements 4 and 5. 
\end{proof}

Let us take structures $G$ and $G'$. Since we can assume the data to be fixed we lighten
up the notations and write 
$P^1_{NML}(i,G_i) := P^1_{NML}(D[\cdot,(i,G_i)];G)$ and
$P^1_{NML}(G_i) := P^1_{NML}(D[\cdot,G_i];G)$.

\begin{theorem}
$  s^{qNML}(D;G)=s^{qNML}(D;G').$
\end{theorem}

\begin{proof}

  Now the scores for structures can be decomposed as the
  $s^{qNML}(D;G)=\sum_{i=1}^{n}s_i^{qNML}(D;G)$ and 
  $s^{qNML}(D;G')=\sum_{i=1}^{n}s_i^{qNML}(D;G')$.

Since only the terms corresponding to the variables $A$ and $B$
in these sums are different, it is enough to show that

$$
s_A^{qNML}(D;G)+s_B^{qNML}(D;G) = s_A^{qNML}(D;G')+s_B^{qNML}(D;G')
$$
Now 

\begin{align*}
s_A^{qNML}(D;G)+s_B^{qNML}(D;G)& =\log\frac{P^1_{NML}(A,G_{A})}{P^1_{NML}(G_{A})}\frac{P^1_{NML}(B,G_{B})}{P^1_{NML}(G_{B})}\\
 & =\log 1\cdot\frac{P^1_{NML}(B,G_{B})}{P^1_{NML}(G_{A})}\\
 & =\log \frac{P^1_{NML}(B,G'_{B})}{P^1_{NML}(G'_{A})}\frac{P^1_{NML}(A,G'_{A})}{P^1_{NML}(G'_{B})}\\
 & =s_A^{qNML}(D;G')+s_B^{qNML}(D;G').
\end{align*}
The second equation follows from the lemma \ref{thm:sameparents}, and the third from
the statements 5 and 4 of the lemma \ref{thm:reveqs}.
\end{proof}
