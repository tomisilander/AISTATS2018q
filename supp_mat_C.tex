\section{qNML coincides with NML for many models}


\subsection{qNML Equals NML for Many Models}
The fNML criterion can be seen as a computationally feasible
approximation of the more desirable NML criterion.  However, the fNML
criterion equals the NML criterion only for the Bayesian network
structure with no arcs.  We will next show that the qNML criterion
equals the NML criterion for all the networks $G$ whose connected
components tournaments (i.e., complete directed acyclic subgraphs of
$G$).

\begin{theorem}
If $G$ consists of $C$ connected components $(G^1,\ldots,G^C)$ with
variable sets $(V^1,\ldots,V^C)$, then $\log P_{NML}(D;G) = s^{qNML}(D;G)$
for all data sets $D$.
\end{theorem}
\begin{proof}
We first show that the NML-criterion for a Bayesian network
decomposes by the connected components.

Because the maximum likelihood for the data $D$ decomposes,
we can write
\begin{eqnarray}
  \lefteqn{P_{NML}(D;G)} \nonumber \\
  && = \frac{P(D;\hat\theta(D),G)}
            {\sum_{D'_{V_1}}\ldots \sum_{D'_{V_C}}\prod_{c=1}^C P(D'_{V^c};\hat\theta(D'_{V^c}),G)} \nonumber \\
            && = \frac{\prod_{c=1}^C P(D_{V^c};\hat\theta(D_{V^c}),G)} 
            {\prod_{c=1}^C\sum_{D'_{V^c}}P(D'_{V^c};\hat\theta(D'_{V^c}),G)}.\nonumber \\
            && = \prod_{c=1}^CP_{NML}(D_{V^c};G).
\end{eqnarray}

Clearly, the qNML score also decomposes by the connected components,
so it remains to show that if the (sub)network $G$ is a tournament,
then for any data $D$, $s^{qNML}(D;G)=\log P_{NML}(D;G)$.  Due to the
score equivalence of the NML criterion and the qNML criterion, we may
pick a tournament $G$ such that the linear ordering defined by $G$
matches the ordering of the data columns, i.e., $i<j$ implies $G_i
\subset G_j$. Now from the definition~(8) of the qNML
criterion (in the main paper), we see that for the tournament $G$, the sum telescopes
leaving us with $s^{qNML}(D;G) = \log P^1_{NML}(D_G;G)$, thus
it is enough to show that $P^1_{NML}(D;G)=P_{NML}(D;G)$.  This
follows, since for any data vector $x$ in data $D$, we have
$P^1(x;\hat\theta(D),G) = P(x;\hat\theta(D),G)$, where $P^1$ denotes
the model that takes $n$-dimensional vectors to be values of the
single (collapsed) categorical variable.  Denoting prefixes of data
vector $x$ by $x^{:i}$, and the number of times such a prefix appears
on $N$ rows $[d_1,\ldots,d_N]$ of the $N\times n$ data matrix $D$ by
$N_D(x^{:i})$, (so that $N_D(x^{:0})=N)$, we have
\begin{eqnarray}
  P(D;\hat\theta(D),G) &=& \prod_{j=1}^{N}P(d_j;\hat\theta(D),G) \nonumber \\
  &=& \prod_{j=1}^{N} \prod_{i=1}^{n} \frac{N_D(d_j^{:i})}{N_D(d_j^{:i-1})} \nonumber\\
  &=& \prod_{j=1}^{N} \frac{N_D(d_j^{:n})}{N} \nonumber \\
  &=& P^1(D;\hat\theta^1(D),G).
\end{eqnarray}
Since both $P^1_{NML}$ and $P_{NML}$ are defined in terms of the
maximum likelihood probabilities, the equality above implies the equality
of these distributions.
\end{proof}

Equality established, we would like to still state the number $a(n)$
of different $n$-node networks whose connected components are
tournaments.  We start by generating all the $p(n)$ integer partitions
i.e. ways to partition $n$ labelled into parts with different
size-profiles.  For example, for $n=4$, we have $p(4)=5$ partition
profiles $[[4], [3, 1], [2, 2], [2, 1, 1], [1, 1, 1, 1]]$. Each of
these partition size profiles corresponds to many different networks,
apart from the last one that corresponds just to the empty network.
We count number of networks for one such partition size-profile (and
then later sum these counts up).  For any such partition profile
$(p_1,\ldots,p_k)$ we can count the ways we can assign the nodes to
different parts and then order each part. This leads to the product
$n\choose p_1$$p_1$!${n-p_1}\choose p_2$$p_2!$${n-p_1-p2}\choose
p_3$$p_3!$$\ldots$${n-\sum_{j=1}^{k-1}p_j}\choose p_k$$p_k!$. However,
the order of different parts of the same size does not matter, so for
all groups of parts having the same size, we have to divide the
product above by the factorial of the size of such group. Notice also
that the product above telescopes, leaving us a formula for OEIS
sequence A000262\footnote{https://oeis.org/A000262} as
described by Thomas Wieder:

\textit{With $p(n) =$ the number of integer partitions of $n$, $d(i)$ = the
number of different parts of the $i^\text{th}$ partition of $n$,
$m(i,j) =$ multiplicity of the $j^\text{th}$ part of the $i^\text{th}$
partition of $n$, one has:}
$$
a(n) = \sum_{i=1}^{p(n)} \frac{n!}{\prod_{j=1}^{d(i)} m(i,j)!}.
$$
For example, $a(4)=\frac{4!}{1!}+\frac{4!}{1!1!}+\frac{4!}{2!}+\frac{4!}{1!2!}+\frac{4!}{4!}=73$. In general this sequence grows rapidly; $1, 1, 3, 13, 73, 501, 4051, 37633, 394353, 4596553, \ldots$.
