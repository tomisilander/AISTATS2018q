\documentclass[12pt]{article}
\usepackage{amsmath}
\usepackage{amsthm}
\usepackage{amssymb}
\usepackage{bm}
\usepackage[round]{natbib}
\usepackage{bm}
\newcommand{\Le}{\left(}
\newcommand{\Ri}{\right)}
\newcommand\independent{\protect\mathpalette{\protect\independenT}{\perp}}
\def\independenT#1#2{\mathrel{\rlap{$#1#2$}\mkern2mu{#1#2}}}
\newtheorem{theorem}{Theorem}
\newtheorem{corollary}{Corollary}
\newtheorem{lemma}{Lemma}


\begin{document}

\section{Problem}

The score function $Q_n(X \mid Y)$, where $X$ refers to one random variable and $Y$ is some proposed parent set for that variable, is \textit{regular} if 
$$H_n(X \mid U) \leq H_n(X \mid U \cup V) \quad \Rightarrow \quad  Q_n(X\mid U) \geq Q_n(X \mid U \cup V), $$ 
where $H_n(\cdot\mid \cdot)$ refers to conditional empirical entropy and $U$ and $V$ are some sets of random variables.

Let $X$ be a categorical random variable with $r$ possible values. Let $U$ denote a possible parent set with $q$ different combinations of values for the variables and $V$ a parent set with $m$ different configurations. Assume that we have observed $n$ samples of $(X,U,V)$ and $H_n(X \mid U) \leq H_n(X \mid U \cup V)$ holds.

We want to show that
$$
\quad  Q^{qnml}_n(X\mid U) \geq Q^{qnml}_n(X \mid U \cup V).
$$
The assumption about entropy implies that the maximized likelihood terms of the qnml-score are equal. In order to prove the claim it suffices to study the penalty terms, and we want to show that
\begin{align*}
-[reg(n,rq) - reg(n,q) ] \ &\geq \  -[reg(n,rqm) - reg(n,qm) ] \\
reg(n,rq) - reg(n,q) \ &\leq \  reg(n,rqm) - reg(n,qm) \\
 \frac{C(n,rq)}{C(n,q)} \ &\leq \frac{C(n,rqm)}{C(n,qm)},
\end{align*} where $C(n,k)$ is the stochastic complexity for multinomial random variable with $k$ categories. Since, trivially, $q \leq qm$, we can prove the claim by showing that the function $k \mapsto C(n,rk)/C(n,k)$ is increasing for arbitrary $r \geq 2$ and $n \geq 1$. 

\section{Stochastic complexity}
We can represent $C(n,k)$ via recursion $$C(n,k) = C(n,k-1) + C(n,k-2)n/(k-2)$$
or by using the following formula
\begin{align*}
C(n,k) &= \sum_{l=0}^n\frac{n^{\underline{l} } \ (k-1)^{\overline{l}}}{n^l \ l!} \\
&= \sum_{l=0}^{n-1}\frac{(n-1)^{\underline{l}}k^{\overline{l + 1}}}{n^{l+1} \ l!} 
\end{align*}where $x^{\underline{l}}$ and $x^{\overline{l}}$ denote falling and rising factorials, respectively.

\section{Case $n = 2$}
\begin{align*}
\frac{C(2,rk)}{C(2,k)} &= \frac{1 + (rk-1) + (rk-1)rk/4}{1 + (k-1) + (k-1)k/4}
= \frac{r^2k^2 + 3rk}{k^2 + 3k},
\end{align*}and since
$$
\frac{d}{dk} \Le\frac{C(2,rk)}{C(2,k)} \Ri =\frac{3r(r-1)k^2}{(k^2+3k)^2} = \frac{3r(r-1)}{(k+3)^2} > 0
$$for every $r \geq 2$ and $k \geq 1$, the claim holds.

\section{Case $n = 3$}
\begin{align*}
\frac{C(3,rk)}{C(3,k)} &= \frac{r^3k^2 + 9r^2k + 17r}{k^2 + 9k + 17} \textnormal{ and } \\ \frac{d}{dk} \Le\frac{C(3,rk)}{C(3,k)}\Ri &= \frac{(r-1)r(153 + 34 (r+1) k + 9 r k^2)}{(17 + 9 k + k^2)^2} > 0
\end{align*}

\section{Case $n = 4$}
\begin{align*}
\frac{C(4,rk)}{C(4,k)} = \frac{142 r + 95 r^2 k + 18 r^3 k^2 + r^4 k^3}{142 + 95 k + 18 k^2 + k^3}
\end{align*}

It seems that the denominator of the derivative contains "always" a polynomial of $k$ with positive coefficients, and since the nominator is some polynomial squared, the resulting quotient is always positive for arbitrary $r$.  
%\section{Case general}
%Using the first sum form from Sec. 2 and denoting $b_l = \frac{n^{\underline{l}}}{n^l l!}$, we have
%\begin{align*}
%\frac{C(n,rk)}{C(n,k)} = \frac{\sum_{l=0}^n b_l(rk-1)^{\overline{l}} } {\sum_{l=0}^n b_l(k-1)^{\overline{l}}}.
%\end{align*}
%
%Differentiating the denominator and nominator  w.r.t. $k$ gives
%\begin{align*}
%\frac{d}{dk}\Le\sum_{l=0}^n b_l(rk-1)^{\overline{l}}\Ri &=  \sum_{l=0}^nr b_l[\psi(rk-i+l)-\psi(rk-1)](rk-1)^{\overline{l}} \\
%&= r  \sum_{l=1}^n b_l\Le\sum_{i=0}^{l-1}\frac{(rk-1)^{\overline{l}}}{rk-1+i}\Ri
%\end{align*}and
%\begin{align*}
%\frac{d}{dk}\Le\sum_{l=0}^n b_l(k-1)^{\overline{l}}\Ri= \sum_{l=1}^n b_l\Le\sum_{i=0}^{l-1}\frac{(k-1)^{\overline{l}}}{k-1+i}\Ri
%\end{align*}
%Differentiating the quotient results in the following expression
%\begin{align*}
%\frac{d}{dk}\Le\frac{C(n,rk)}{C(n,k)}\Ri &= \\
%&r\Le   \sum_{l=1}^n b_l\Le\sum_{i=0}^{l-1}\frac{(rk-1)^{\overline{l}}}{rk-1+i}\Ri \Ri \Le \sum_{l=0}^n b_l(k-1)^{\overline{l}} \Ri \\ 
%-& \Le \sum_{l=1}^n b_l\Le\sum_{i=0}^{l-1}\frac{(k-1)^{\overline{l}}}{k-1+i}\Ri \Ri \Le \sum_{l=0}^n b_l(rk-1)^{\overline{l}}  \Ri 
%\end{align*}Hard to see anything from this..
\section{Case general}
We first derive a representation for $C(n,k)$ as a polynomial of $k$. We utilize the fact that the rising factorial can be represented as polynomial using unsigned Stirling numbers of the first kind
\begin{align*}
C(n,k) &=  \sum_{l=0}^{n-1}\frac{(n-1)^{\underline{l}}k^{\overline{l + 1}}}{n^{l+1} \ l!} \\
&= \sum_{l=0}^{n-1} b_l \ k^{\overline{l+1}} \\
&= \sum_{l=0}^{n-1} b_l \Le \sum_{j=1}^{l+1}|s(l + 1,j)| \ k^j \Ri \\
&= \sum_{l=0}^{n-1} \Le \sum_{j=1}^{n}b_l \ |s(l + 1,j)| \ k^j \Ri \\
&= \sum_{j=1}^{n} \Le \sum_{l=0}^{n-1} b_l \ |s(l + 1,j)| \ k^j \Ri \\
&= \sum_{j=1}^{n} \Le \sum_{l=0}^{n-1} b_l \ |s(l + 1,j)| \Ri k^j \\
&= \sum_{j=1}^{n} a_j  k^j,
\end{align*}where $s(x,y)$ denotes the Stirling number of the first kind and
$$
a_j = \Le \sum_{l=0}^{n-1} \frac{(n-1)^{\underline{l}}}{n^{l+1}l!} \ |s(l + 1,j)| \Ri,
$$ $a_j \geq 0$ for all $j$ (also it seems that $\sum_{j=1}^n a_j = 1 $). On the row 4, we used the property of Stirling numbers: $s(i,j) = 0$ for all $j > i$. Similarly,
$$
C(n,rk) = \sum_{j=1}^{n} a_j  r^jk^j
$$Derivatives are obtained easily from this form
\begin{align*}
\frac{d}{dk}C(n,k) &= \sum_{j=1}^{n}ja_jk^{j-1} \\
&= \sum_{j=0}^{n-1}(j+1) a_{j+1}k^{j}
\end{align*}and
\begin{align*}
\frac{d}{dk}C(n,rk) &= \sum_{j=1}^{n}ja_jr^jk^{j-1} \\
&= \sum_{j=0}^{n-1}(j+1)a_{j+1}r^{j+1}k^{j}. 
\end{align*}Consider next the products found in the derivative of the quotient. We obtain
\begin{align*}
\Le\frac{d}{dk}C(n,rk)\Ri C(n,k) &= \Le \sum_{j=0}^{n-1}(j+1)a_{j+1}r^{j+1}k^{j} \Ri \Le \sum_{l=1}^{n} a_l  k^l \Ri \\
&= \sum_{i = 1}^{2n-1} \Le \sum_{j+l = i} (j+1)a_{j+1}r^{j+1}a_l  \Ri k^i
\end{align*}and
%\begin{align*}
%\Le\frac{d}{dk}C(n,k)\Ri C(n,rk) &= \Le \sum_{j=0}^{n-1}(j+1)a_{j+1}k^{j} \Ri \Le %\sum_{l=1}^{n} a_l  r^lk^l  \Ri \\
%&= \sum_{i = 1}^{2n-1} \Le \sum_{m=0}^i (m+1)a_{m+1}a_{i-m}r^{i-m}   \Ri k^i
%\end{align*}
\begin{align*}
\Le\frac{d}{dk}C(n,k)\Ri C(n,rk) &= \Le \sum_{j=0}^{n-1}(j+1)a_{j+1}k^{j} \Ri \Le \sum_{l=1}^{n} a_l  r^lk^l  \Ri \\
&= \sum_{i = 1}^{2n-1} \Le \sum_{j+l = i}(j+1)a_{j+1}a_l  r^l   \Ri k^i.
\end{align*}
Subtracting these two expression yields
\begin{align*}
&\Le\frac{d}{dk}C(n,rk)\Ri C(n,k)-\Le\frac{d}{dk}C(n,k)\Ri C(n,rk) \\&= \sum_{i = 1}^{2n-1} \Le \sum_{j+l = i} (j+1)a_{j+1}r^{j+1}a_l  \Ri k^i - \sum_{i = 1}^{2n-1} \Le \sum_{j+l = i}(j+1)a_{j+1}a_l  r^l   \Ri k^i \\
&= \sum_{i = 1}^{2n-1} \Le \sum_{j+l = i}(j+1)a_{j+1}a_l  (r^{j+1}- r^l)   \Ri k^i
\end{align*}which is the polynomial in the denominator of the derivative of $C(n,rk)/C(n,k)$.
Next, we study the coefficient of $k^i$, if $i \leq n$
\begin{align*}
\sum_{j+l = i}(j+1)a_{j+1}a_l  (r^{j+1}- r^l) &= \sum_{l=1}^{i}(i-l+1)a_{i-l+1}a_l  (r^{i-l+1}- r^l) \\
& = \sum_{l=1}^{i}(i-l+1)c_l \\
&= \sum_{k =1}^{\lfloor i / 2 \rfloor}(i-k+1)c_k + (i-(i-k + 1)+1)c_{i-k+1} \\
&=  \sum_{k =1}^{\lfloor i / 2 \rfloor}(i-k+1)c_k + kc_{i-k+1} \\
&=  \sum_{k =1}^{\lfloor i / 2 \rfloor}(i-k+1)c_k - kc_{k} \\
&= \sum_{k =1}^{\lfloor i / 2 \rfloor}(i-2k+1)c_k.
\end{align*} On the first row, we re-wrote sum using only one running index. On the second row we denoted $c_l =a_{i-l+1}a_l  (r^{i-l+1}- r^l)$. On the third row, we re-arranged the sum so that we are summing over pairs of terms of the original sum: the first and the last term, the second and the second to last, and so on.  This resulting sum has $\lfloor i / 2 \rfloor$ terms. We have to use the floor-function since if $i$ is odd, there exists an index $l'$ in the original sum such that $r^{i-l'+1}-r^{l'} = 0$. On the fifth row, we make use of the identity $c_k = -c_{i-k+1}$ which is straightforward to verify. From the last row, we can observe that every term of the sum is positive since $i-2k+1$ and $r^{i-k+1}- r^k$ are both positive if $k \leq (i+1)/2$ which holds since $k$ ranges from $1$ to $\lfloor i / 2 \rfloor$.

Let us now consider the situation where $n < i \leq 2n-1$. We start with the special case where $i = 2n-1$. Then, we have only one term in the sum
\begin{align*}
\sum_{j+l = i}(j+1)a_{j+1}a_l  (r^{j+1}- r^l) &= \sum_{l=n}^{n}(2n-1-l+1)a_{2n-1-l+1}a_l  (r^{2n-1-l+1}- r^l) \\
&= na_{n}a_n  (r^{n}- r^n) \\
& = 0.
\end{align*}Now, let $n < i < 2n-1$, we follow a similar procedure as before to manipulate the sum
\begin{align*}
\sum_{j+l = i}(j+1)a_{j+1}a_l  (r^{j+1}- r^l) &= \sum_{l=i-n + 1}^{n}(i-l+1)a_{i-l+1}a_l  (r^{i-l+1}- r^l) \\
& = \sum_{l=i-n + 1}^{n}(i-l+1)c_l \\
& = \sum_{k=1}^{2n-i}(n-k+1)c_{i-n+k} \\
& = \sum_{k=1}^{\lfloor n - i/2 \rfloor}(n-k+1)c_{i-n+k}\\
&+  (n-(2n-i-k+1)+1)c_{i-n+(2n-i-k+1)} \\
&= \sum_{k=1}^{\lfloor n - i/2 \rfloor}(n-k+1)c_{i-n+k} +  (i-n+k)c_{n-k+1} \\
&= \sum_{k=1}^{\lfloor n - i/2 \rfloor}(n-k+1)c_{i-n+k} -  (i-n+k)c_{i-n+k} \\
&= \sum_{k=1}^{\lfloor n - i/2 \rfloor}(n-k+1-(i-n+k))c_{i-n+k} \\
&= \sum_{k=1}^{\lfloor n - i/2 \rfloor}(2n-i-2k+1))c_{i-n+k}.
\end{align*}It is now easy to verify that $(2n-i-2k+1)$ and $c_{i-n+k}$ are positive if $k \leq n - (i+2)/2$ which holds since $k$ ranges from $1$ to $\lfloor n - i/2 \rfloor$. The floor function is again used when we sum over pairs of terms since if $i$ is odd there is zero-term. 
Since all the coefficients are non-negative and the $k \geq 2$, the derivative is positive. This implies that the original function is increasing.


%$$
%C(n,rk) = \sum_{l=0}^{n-1} b_l \Le \sum_{j=1}^{l+1}|s(l + 1,j)| \ (rk)^j \Ri.
%$$ Derivatives are obtained easily from this form
%$$&=  \sum_{k=i-n + 1}^{\lfloor n - i/2 \rfloor}(i-l+1)c_k + (i-(n-k + 1)+1)c_{n-k+1} 
%\frac{d}{dk}C(n,k) = \sum_{l=0}^{n-1} b_l \Le\sum_{j=1}^{l+1}j \ |s(l + 1,j)| \ k^{j-1}\Ri
%$$and
%$$
%\frac{d}{dk}C(n,rk) = \sum_{l=0}^{n-1} b_l \Le \sum_{j=1}^{l+1}j \ r^{j} \ |s(l + 1,j)| \ k^{j-1}\Ri.
%$$
\end{document}