\documentclass[12pt]{article}
\usepackage{amsmath}
\usepackage{amsthm}
\usepackage{amssymb}
\usepackage{bm}
\usepackage[round]{natbib}
\usepackage{bm}
\newcommand{\Le}{\left(}
\newcommand{\Ri}{\right)}
\newcommand{\N}{\mathbb{N}}
\newcommand\independent{\protect\mathpalette{\protect\independenT}{\perp}}
\def\independenT#1#2{\mathrel{\rlap{$#1#2$}\mkern2mu{#1#2}}}
\newtheorem{theorem}{Theorem}
\newtheorem{definition}{Definition}
\newtheorem{corollary}{Corollary}
\newtheorem{lemma}{Lemma}

\date{}
\title{Supplementary Material for the article \textit{Quotient Normalized Maximum Likelihood Criterion for Learning Bayesian Network Structures} }
\begin{document}
\maketitle

\appendix
\section{Regularity proof}

\subsection{Preliminaries}
We start by recalling the definition of regularity \citep{Suzuki2017}:

\begin{definition}
Assume $H_N(X \mid Y') \leq H_N(X \mid Y)$, where $Y' \subset Y.$ We say that the scoring function $Q_N(\cdot \mid \cdot)$ is regular if $Q_n(X \mid Y') \geq Q_N(X \mid Y)$.
\end{definition}
In the definition, $N$ denotes the sample size, $X$ is some random variable, $Y$ denotes the proposed parent set for $X$, and $H_N(X \mid Y)$ refers to the empirical conditional entropy based on $N$ samples of variables $X$ and $Y$.

Let $X$ be a categorical random variable with $r$ possible values. Let $U$ denote a possible parent set with $q$ different combinations of values for the variables, and $V$ a parent set with $m$ different configurations. Assume that we have observed $N$ samples of $(X,U,V)$ (denoted by $x_N,u_N$ and $v_N$) and $H_N(X \mid U) \leq H_N(X \mid U \cup V)$ holds.

Recall the definition of the qNML score:
\begin{align*}
Q^{qnml}_N(X\mid U) &= \log P(x_n \mid \hat{\theta}_{X\mid U} \ ) - \Le reg(N,rq) - reg(N,q) \Ri \\
&= \log P(x_N \mid \hat{\theta}_{X\mid U} \ ) - \log \frac{C(N,rq)}{C(N,q)},
\end{align*}where $C(N,r)$ is the normalizing constant the of the NML distribution for a categorical variable with $r$ possible values and sample size $N$ and $ \hat{\theta}_{X\mid U}$ denotes the maximum likelihood parameters of the conditional distribution of $X$ given $U$ which are computed from the data $(x_N,u_N)$. 


In order to prove the regularity, we need the following three lemmas:
\begin{lemma}\label{Cpolynom} We can write $C(N,k)$ as a polynomial of $k$, formally
$$
C(N,k) = \sum_{j=1}^N a_j k^j,
$$where $a_j > 0$. 
\end{lemma}
\begin{lemma}\label{MLterms}
Assume $H_N(X \mid Y') \leq H_N(X \mid Y)$, where $Y' \subset Y$. Now $\log P(x_N \mid \hat{\theta}_{X\mid Y} \ ) = \log P(x_N \mid \hat{\theta}_{X\mid Y'} \ )$.
\end{lemma}
\begin{lemma}\label{increasing}
Let $r \in \N, r \geq 2$.  The function $k \mapsto \frac{C(N,rk)}{C(N,k)}$ is increasing for every $k \geq 2$.
\end{lemma}
\noindent We present the proofs of these lemmas in Section \ref{lemmaproofs}.  

\subsection{The main proof}
\begin{theorem}
qNML score is regular.
\end{theorem}
\begin{proof}
We want to show that
$$
Q^{qnml}_N(X\mid U) \geq Q^{qnml}_N(X \mid U \cup V).
$$assuming $H_N(X \mid U) \leq H_N(X \mid U \cup V)$. Using the entropy assumption and Lemma \ref{MLterms} implies that the maximized likelihood terms are equal. In order to prove the claim, it suffices to study the penalty terms, and we want to show that
\begin{align*}
- \Le reg(N,rq) - reg(N,q) \Ri \ &\geq \  -\Le reg(N,rqm) - reg(N,qm) \Ri \\
\log \frac{C(N,rq)}{C(N,q)} \ &\leq \log \frac{C(N,rqm)}{C(N,qm)}.
\end{align*}This holds, since logarithm is an increasing function, and $q \leq qm$, so we can apply Lemma \ref{increasing} to conclude the proof. 


\end{proof}


\subsection{Proofs of lemmas}\label{lemmaproofs}
%We want to show that
%$$
%\quad  Q^{qnml}_N(X\mid U) \geq Q^{qnml}_N(X \mid U \cup V)
%$$which, by using the definition of qNML, is equivalent to 
%\begin{equation}\label{statement}
%\log P(X \mid )
%\end{equation}
%The assumption about entropy implies that the maximized likelihood terms of the qnml-score are equal. 
\setcounter{lemma}{0}
\begin{lemma} $C(N,k)$ can be written as a polynomial of $k$, formally
$$
C(N,k) = \sum_{j=1}^N a_j k^j,
$$where $a_j > 0$. 
\end{lemma}
\begin{proof} \cite{cosco.itsl08} derive the following representation for the normalizing constant
\begin{align*}  
C(N,k) = \sum_{l=0}^{N-1}\frac{(N-1)^{\underline{l}}k^{\overline{l + 1}}}{N^{l+1} \ l!} ,
\end{align*}where $x^{\underline{l}}$ and $x^{\overline{l}}$ denote falling and rising factorials, respectively. 

We utilize the fact that the rising factorial can be represented as polynomial using unsigned Stirling numbers of the first kind (see \cite{Adamchik1997}, for instance)
\begin{align*}
C(N,k) &=  \sum_{l=0}^{N-1}\frac{(N-1)^{\underline{l}}k^{\overline{l + 1}}}{N^{l+1} \ l!} \\
&= \sum_{l=0}^{N-1} b_l \ k^{\overline{l+1}} \\
&= \sum_{l=0}^{N-1} b_l \Le \sum_{j=1}^{l+1}|s(l + 1,j)| \ k^j \Ri \\
&= \sum_{l=0}^{N-1} \Le \sum_{j=1}^{N}b_l \ |s(l + 1,j)| \ k^j \Ri \\
&= \sum_{j=1}^{N} \Le \sum_{l=0}^{N-1} b_l \ |s(l + 1,j)| \ k^j \Ri \\
&= \sum_{j=1}^{N} \Le \sum_{l=0}^{N-1} b_l \ |s(l + 1,j)| \Ri k^j \\
&= \sum_{j=1}^{N} a_j  k^j,
\end{align*}where $s(i,j)$ denotes the (signed) Stirling number of the first kind and
$$
a_j = \Le \sum_{l=0}^{N-1} \frac{(N-1)^{\underline{l}}}{N^{l+1}l!} \ |s(l + 1,j)| \Ri,
$$ $a_j > 0$ for all $j$. On the second row, we denoted $b_l = (N-1)^{\underline{l}}/(N^{l+1} l!)$. On the row 4, we used the property of Stirling numbers: $s(i,j) = 0$ for all $j > i$. 
\end{proof}
\begin{lemma}
Assume $H_N(X \mid Y') \leq H_N(X \mid Y)$, where $Y' \subset Y$. Now $\log P(x_N \mid \hat{\theta}_{X\mid Y} \ ) = \log P(x_N \mid \hat{\theta}_{X\mid Y'})$.
\end{lemma}
\begin{proof}
We can write the logarithm of the maximized likelihood, \\ $\log P(x_N \mid \hat{\theta}_{X\mid Y} \ )$, as follows \citep{KOLLER}
\begin{align*}
\log P(x_N \mid \hat{\theta}_{X\mid Y} \ ) & = -N \Le H_N(X) - I_N(X;Y) \Ri \\
&= -N \  H_N(X\mid Y),
\end{align*}where $I_N(\cdot;\cdot)$ is the empirical mutual information. This implies that the assumption
$$
H_N(X \mid Y') \leq H_N(X \mid Y)
$$ is equivalent to
$$
\log P(x_N \mid \hat{\theta}_{X\mid Y} \ ) \geq \log P(x_N \mid \hat{\theta}_{X\mid Y'} \ ) .
$$Actually we must have the equality holding in the above expression, since
$$
H_N(X \mid Y') < H_N(X \mid Y)
$$ would imply that
$$
I_N(X ; Z \mid Y ) < 0,
$$where $Z = Y \setminus Y'$, which is impossible.
\end{proof}
\begin{lemma}
Let $r \in \N, r \geq 2$.  The function $k \mapsto \frac{C(N,rk)}{C(N,k)}$ is increasing for every $k \geq 2$.
\end{lemma}

\begin{proof}
Lemma \ref{Cpolynom} lets us to write
\begin{equation}\label{C1}
C(N,k) = \sum_{j=1}^{N} a_j k^j
\end{equation}and, similarly,
\begin{equation}\label{C2}
C(N,rk) = \sum_{j=1}^{N} a_j  r^jk^j.
\end{equation}

From this, it it easy see that the derivative of the quotient, \\ $d/dk (C(N,rk)/C(N,k))$, will be a ratio of two polynomials of $k$. Our goal is to show that the polynomial in the numerator has positive coefficients, which will guarantee the positivity of derivative for every $k > 0$, and thus imply that the original function is increasing (polynomial in the denominator is squared and non-zero for $k > 0$, so it can be ignored).  

Derivatives of (\ref{C1}) and (\ref{C2}) are obtained easily:
\begin{align*}
\frac{d}{dk}C(N,k) &= \sum_{j=1}^{N}ja_jk^{j-1} \\
&= \sum_{j=0}^{N-1}(j+1) a_{j+1}k^{j}
\end{align*}and
\begin{align*}
\frac{d}{dk}C(N,rk) &= \sum_{j=1}^{N}ja_jr^jk^{j-1} \\
&= \sum_{j=0}^{N-1}(j+1)a_{j+1}r^{j+1}k^{j}. 
\end{align*}Consider next the products found in the derivative of the quotient. We obtain
\begin{align*}
\Le\frac{d}{dk}C(N,rk)\Ri C(N,k) &= \Le \sum_{j=0}^{N-1}(j+1)a_{j+1}r^{j+1}k^{j} \Ri \Le \sum_{l=1}^{N} a_l  k^l \Ri \\
&= \sum_{i = 1}^{2N-1} \Le \sum_{j+l = i} (j+1)a_{j+1}r^{j+1}a_l  \Ri k^i
\end{align*}and
%\begin{align*}
%\Le\frac{d}{dk}C(n,k)\Ri C(n,rk) &= \Le \sum_{j=0}^{n-1}(j+1)a_{j+1}k^{j} \Ri \Le %\sum_{l=1}^{n} a_l  r^lk^l  \Ri \\
%&= \sum_{i = 1}^{2n-1} \Le \sum_{m=0}^i (m+1)a_{m+1}a_{i-m}r^{i-m}   \Ri k^i
%\end{align*}
\begin{align*}
\Le\frac{d}{dk}C(N,k)\Ri C(N,rk) &= \Le \sum_{j=0}^{N-1}(j+1)a_{j+1}k^{j} \Ri \Le \sum_{l=1}^{N} a_l  r^lk^l  \Ri \\
&= \sum_{i = 1}^{2N-1} \Le \sum_{j+l = i}(j+1)a_{j+1}a_l  r^l   \Ri k^i.
\end{align*}
Subtracting these two expression yields
\begin{align*}
&\Le\frac{d}{dk}C(N,rk)\Ri C(N,k)-\Le\frac{d}{dk}C(N,k)\Ri C(N,rk) \\&= \sum_{i = 1}^{2N-1} \Le \sum_{j+l = i} (j+1)a_{j+1}r^{j+1}a_l  \Ri k^i - \sum_{i = 1}^{2N-1} \Le \sum_{j+l = i}(j+1)a_{j+1}a_l  r^l   \Ri k^i \\
&= \sum_{i = 1}^{2N-1} \Le \sum_{j+l = i}(j+1)a_{j+1}a_l  (r^{j+1}- r^l)   \Ri k^i
\end{align*}which is the polynomial in the numerator of the derivative of $C(N,rk)/C(N,k)$.
Next, we study the coefficient of $k^i$, if $i \leq N$
\begin{align*}
\sum_{j+l = i}(j+1)a_{j+1}a_l  (r^{j+1}- r^l) &= \sum_{l=1}^{i}(i-l+1)a_{i-l+1}a_l  (r^{i-l+1}- r^l) \\
& = \sum_{l=1}^{i}(i-l+1)c_l \\
&= \sum_{k =1}^{\lfloor i / 2 \rfloor}(i-k+1)c_k + (i-(i-k + 1)+1)c_{i-k+1} \\
&=  \sum_{k =1}^{\lfloor i / 2 \rfloor}(i-k+1)c_k + kc_{i-k+1} \\
&=  \sum_{k =1}^{\lfloor i / 2 \rfloor}(i-k+1)c_k - kc_{k} \\
&= \sum_{k =1}^{\lfloor i / 2 \rfloor}(i-2k+1)c_k.
\end{align*} On the first row, we re-wrote sum using only one running index. On the second row we denoted $c_l =a_{i-l+1}a_l  (r^{i-l+1}- r^l)$. On the third row, we re-arranged the sum so that we are summing over pairs of terms of the original sum: the first and the last term, the second and the second to last, and so on.  This resulting sum has $\lfloor i / 2 \rfloor$ terms. We have to use the floor-function since if $i$ is odd, there exists an index $l'$ in the original sum such that $r^{i-l'+1}-r^{l'} = 0$. On the fifth row, we make use of the identity $c_k = -c_{i-k+1}$ which is straightforward to verify. From the last row, we can observe that every term of the sum is positive since $i-2k+1$ and $r^{i-k+1}- r^k$ are both positive if $k \leq (i+1)/2$ which holds since $k$ ranges from $1$ to $\lfloor i / 2 \rfloor$.

Let us now consider the situation where $n < i \leq 2N-1$. We start with the special case where $i = 2N-1$. Then, we have only one term in the sum
\begin{align*}
\sum_{j+l = i}(j+1)a_{j+1}a_l  (r^{j+1}- r^l) &= \sum_{l=N}^{N}(2N-1-l+1)a_{2N-1-l+1}a_l  (r^{2N-1-l+1}- r^l) \\
&= Na_{N}a_N  (r^{N}- r^N) \\
& = 0.
\end{align*}Now, let $N < i < 2N-1$, we follow a similar procedure as before to manipulate the sum
\begin{align*}
\sum_{j+l = i}(j+1)a_{j+1}a_l  (r^{j+1}- r^l) &= \sum_{l=i-N + 1}^{N}(i-l+1)a_{i-l+1}a_l  (r^{i-l+1}- r^l) \\
& = \sum_{l=i-N + 1}^{N}(i-l+1)c_l \\
& = \sum_{k=1}^{2N-i}(N-k+1)c_{i-N+k} \\
& = \sum_{k=1}^{\lfloor N - i/2 \rfloor}(N-k+1)c_{i-N+k}\\
&+  (N-(2N-i-k+1)+1)c_{i-N+(2N-i-k+1)} \\
&= \sum_{k=1}^{\lfloor N - i/2 \rfloor}(N-k+1)c_{i-N+k} +  (i-N+k)c_{N-k+1} \\
&= \sum_{k=1}^{\lfloor N - i/2 \rfloor}(N-k+1)c_{i-N+k} -  (i-N+k)c_{i-N+k} \\
&= \sum_{k=1}^{\lfloor N - i/2 \rfloor}(N-k+1-(i-N+k))c_{i-N+k} \\
&= \sum_{k=1}^{\lfloor N - i/2 \rfloor}(2N-i-2k+1)c_{i-N+k}.
\end{align*}It is now easy to verify that $(2N-i-2k+1)$ and $c_{i-N+k}$ are positive if $k \leq N - (i-1)/2$ which holds since $k$ ranges from $1$ to $\lfloor N - i/2 \rfloor$. The floor function is again used when we sum over pairs of terms since if $i$ is odd there is zero-term. 
Since all the coefficients are non-negative and the $k \geq 2$, the derivative is positive. This implies that the original function is increasing.
\end{proof}
\bibliographystyle{apalike}
\bibliography{cosco}
\end{document}